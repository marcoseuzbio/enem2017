%%%%%%%%%%%%%%%%% NÃO EDITAR %%%%%%%%%%%%%%%%%%%%%%%%%%%%%%%%%%%%%%%%%%%%%%%
\documentclass[12pt,openright,twoside,a4paper,chapter=TITLE,english,french,spanish,brazil]{abntex2}
\usepackage{cmap}			
\usepackage{lmodern}	
%\usepackage{abntex2cite}
\usepackage{graphicx}	
% o pacote threeparttable é usado  para o tablenotes, ou seja a nota das tabelas
\usepackage{threeparttable}
\usepackage[T1]{fontenc}		
\usepackage[utf8]{inputenc}	
\usepackage{lastpage}			
\usepackage{indentfirst}	
\usepackage{color}				
\usepackage{graphicx}			
\usepackage{pacotence-capa-folhaprovacao}
%\usepackage{lipsum}			
\usepackage[brazilian,hyperpageref]{backref}
\usepackage[alf]{abntex2cite}

\renewcommand{\backrefpagesname}{Citado na(s) página(s):~}
\renewcommand{\backref}{}
\renewcommand*{\backrefalt}[4]{
	\ifcase #1 %
		Nenhuma citação no texto.%
	\or
		Citado na página #2.%
	\else
		Citado #1 vezes nas páginas #2.%
	\fi}%
	
%%%%%%%%%%%%%%%%%%%%%%%%%%%%%%%%%%%%%%%% EDITAR AQUI %%%%%%%%%%%%%%%%%%%%%%%%%%%%%%%%%

\titulo{Analise nas variáveis socioeconômicas o ENEM em 2017: que influenciam na aprovação dos candidatos inscritos} % O ALUNO DEVE EDITAR AQUI
\autor{Marcos Antonio Euzebio de Oliveira}           % O ALUNO DEVE EDITAR AQUI
\local{Rio de Janeiro}     % O ALUNO DEVE EDITAR AQUI
\data{2019}										% O ALUNO DEVE EDITAR AQUI
\orientador[Orientador(a):]{Alinne de Carvalho Veiga} % O ALUNO DEVE EDITAR AQUI
%\coorientador[Coorientador(a):]{Professor(a)} % O ALUNO DEVE EDITAR AQUI
%%%%%%%%%%%%%%%%%%%%%%%%%%%%%%%%%%%%%%%% NÃO EDITAR %%%%%%%%%%%%%%%%%%%%%%%%%%%%%%%%%%%%%%%

\instituicao{%
  Instituto Brasileiro de Geografia e Estatística --- IBGE
  \par
  Escola Nacional de Ciências Estatísticas --- ENCE
  \par
  Bacharelado em Estatística}
\tipotrabalho{Monografia}
\preambulo{Monografia apresentada à Escola Nacional de Ciências Estatísticas do Instituto Brasileiro de Geografia e Estatística, como requisito parcial à obtenção do título de Bacharel em Estatística.}
\definecolor{blue}{RGB}{41,5,195}
\makeatletter
\hypersetup{
		pdftitle={\@title}, 
		pdfauthor={\@author},
    	pdfsubject={\imprimirpreambulo},
	    pdfcreator={LaTeX with abnTeX2},
		pdfkeywords={abnt}{latex}{abntex}{abntex2}{trabalho acadêmico}, 
		colorlinks=true,       		
    	linkcolor=blue,          
    	citecolor=blue,        	
    	filecolor=magenta,      		
		urlcolor=blue,
		bookmarksdepth=4
}

\makeatother

\setlength{\parindent}{1.3cm}

\setlength{\parskip}{0.2cm} 

\makeindex

\begin{document}
%
\frenchspacing 
%
\imprimircapa
%
\imprimirfolhaderosto*
% ---
% Inserir a ficha bibliografica
% ---
% Isto é um exemplo de Ficha Catalográfica, ou ``Dados internacionais de
% catalogação-na-publicação''. Você pode utilizar este modelo como referência. 
% Porém, provavelmente a biblioteca da sua universidade lhe fornecerá um PDF
% com a ficha catalográfica definitiva após a defesa do trabalho. Quando estiver
% com o documento, salve-o como PDF no diretório do seu projeto e substitua todo
% o conteúdo de implementação deste arquivo pelo comando abaixo:
%
% \begin{fichacatalografica}
%     \includepdf{fig_ficha_catalografica.pdf}
% \end{fichacatalografica}

%\begin{fichacatalografica}
%	\vspace*{\fill}				
%	\hrule							
%	\begin{center}					
%	\begin{minipage}[c]{12.5cm}		
%	
%	\imprimirautor
%	l
%	\hspace{0.5cm} \imprimirtitulo  / \imprimirautor. --
%	\imprimirlocal, \imprimirdata-
%	
%	\hspace{0.5cm} \pageref{LastPage} p. : il. (algumas color.) ; 30 cm.\\
%	
%	\hspace{0.5cm} \imprimirorientadorRotulo~\imprimirorientador\\
%	
%	\hspace{0.5cm}
%	\parbox[t]{\textwidth}{\imprimirtipotrabalho~--~\imprimirinstituicao,
%	\imprimirdata.}\\
%	
%	\hspace{0.5cm}
%		1. Palavra-chave1.
%		2. Palavra-chave2.
%		I. Orientador.
%		II. Universidade xxx.
%		III. Faculdade de xxx.
%		IV. Título\\ 			
%	
%	\hspace{8.75cm} CDU 02:141:005.7\\
%	
%	\end{minipage}
%	\end{center}
%	\hrule
%\end{fichacatalografica}

%%%%%%%%%%%%%%%%%%%%%%%%%%%%%%%%%%%%% O ALUNO DEVE EDITAR AQUI %%%%%%%%%%%%%%%%%%%%%%%%%%%%%%%%%%%%%%%%

%\begin{errata}
%Elemento opcional da \citeonline[4.2.1.2]{NBR14724:2011}. Exemplo:
%
%\vspace{\onelineskip}
%
%FERRIGNO, C. R. A. \textbf{Tratamento de neoplasias ósseas apendiculares com
%reimplantação de enxerto ósseo autólogo autoclavado associado ao plasma
%rico em plaquetas}: estudo crítico na cirurgia de preservação de membro em
%cães. 2011. 128 f. Tese (Livre-Docência) - Faculdade de Medicina Veterinária e
%Zootecnia, Universidade de São Paulo, São Paulo, 2011.
%
%\begin{table}[htb]
%\center
%\footnotesize
%\begin{tabular}{|p{1.4cm}|p{1cm}|p{3cm}|p{3cm}|}
%  \hline
%   \textbf{Folha} & \textbf{Linha}  & \textbf{Onde se lê}  & \textbf{Leia-se}  \\
%    \hline
%    1 & 10 & auto-conclavo & autoconclavo\\
%   \hline
%\end{tabular}
%\end{table}
%
%\end{errata}
%
%%%%%%%%%%%%%%%%%%%%%%%%%%%%%% O ALUNO DEVE EDITAR ONDE ESTAR MARCADO %%%%%%%%%%%%%%%%%%%%%%%%%
%
%\begin{folhadeaprovacao}
%
%  \begin{center}
%    {\ABNTEXchapterfont\large\imprimirautor}
%
%    \vspace*{\fill}\vspace*{\fill}
%    {\ABNTEXchapterfont\bfseries\Large\imprimirtitulo}
%    \vspace*{\fill}
%		
%    \hspace{.45\textwidth}
%    \begin{flushleft}
%        \imprimirpreambulo
%   \end{flushleft}%
%	\vspace*{\fill}
%	
%	\begin{flushleft}
%	 Aprovada em 24 de dezembro de 2014.			% O ALUNO DEVE EDITAR AQUI
%	   \end{flushleft}%
%		
%    \vspace*{\fill}
%		 {\ABNTEXchapterfont\bfseries\large BANCA EXAMINADORA}
%		\vspace*{\fill}
%   \end{center}
%    
%   \assinatura{\textbf{\imprimirorientador} \\ Orientador(a) - SIGLA }   % O ALUNO DEVE EDITAR AQUI
%   \assinatura{\textbf{\imprimircoorientador} \\ Coorientador(a) - SIGLA}  % O ALUNO DEVE EDITAR AQUI
%   \assinatura{\textbf{Professor} \\ Convidado(a) - SIGLA}         % O ALUNO DEVE EDITAR AQUI
%   %\assinatura{\textbf{Professor} \\ Convidado 3}
%   %\assinatura{\textbf{Professor} \\ Convidado 4}
%      
%   \begin{center}
%    \vspace*{0.5cm}
%    {\imprimirlocal}
%    \par
%    {\imprimirdata}
%    \vspace*{1cm}
%  \end{center}
%  
%\end{folhadeaprovacao}
%
%%%%%%%%%%%%%%%%%%%%%%%%%%%%%%%%%%%%%%%% O ALUNO DEVE EDITAR ONDE ESTAR MARCADO %%%%%%%%%%%%%%%%%%%%%%%%%%%%%%%%%%%%%%%%%
%
%\begin{dedicatoria}
%   \vspace*{\fill}
%   \centering
%   \noindent
%   \textit{ Este trabalho é dedicado às crianças adultas que,\\   % O ALUNO DEVE EDITAR AQUI
%   quando pequenas, sonharam em se tornar cientistas.} \vspace*{\fill}   % O ALUNO DEVE EDITAR AQUI
%\end{dedicatoria}
%
%%%%%%%%%%%%%%%%%%%%%%%%%%%%% ALUNO DEVE EDITAR AQUI %%%%%%%%%%%%%%%%%%%%%%%%%%%%%%%%%%%%%%%%%%%%%
%
%\begin{agradecimentos}    % O ALUNO DEVE EDITAR AQUI
%Os agradecimentos principais são direcionados à Gerald Weber, Miguel Frasson,
%Leslie H. Watter, Bruno Parente Lima, Flávio de Vasconcellos Corrêa, Otavio Real
%Salvador, Renato Machnievscz\footnote{Os nomes dos integrantes do primeiro
%projeto abn\TeX\ foram extraídos de
%\url{http://codigolivre.org.br/projects/abntex/}} e todos aqueles que
%contribuíram para que a produção de trabalhos acadêmicos conforme
%as normas ABNT com \LaTeX\ fosse possível.
%
%Agradecimentos especiais são direcionados ao Centro de Pesquisa em Arquitetura
%da Informação\footnote{\url{http://www.cpai.unb.br/}} da Universidade de
%Brasília (CPAI), ao grupo de usuários
%\emph{latex-br}\footnote{\url{http://groups.google.com/group/latex-br}} e aos
%novos voluntários do grupo
%\emph{\abnTeX}\footnote{\url{http://groups.google.com/group/abntex2} e
%\url{http://abntex2.googlecode.com/}}~que contribuíram e que ainda
%contribuirão para a evolução do \abnTeX.
%
%\end{agradecimentos}
%%%%%%%%%%%%%%%%%%%%%%%%%%%%%% ALUNO DEVE EDITAR AQUI %%%%%%%%%%%%%%%%%%%%%%%%%%%%%%%%%%%
%\begin{epigrafe}
%    \vspace*{\fill}
%	\begin{flushright}  % O ALUNO DEVE EDITAR AQUI
%		\textit{``Não vos amoldeis às estruturas deste mundo, \\
%		mas transformai-vos pela renovação da mente, \\
%		a fim de distinguir qual é a vontade de Deus: \\
%		o que é bom, o que Lhe é agradável, o que é perfeito.\\
%		(Bíblia Sagrada, Romanos 12, 2)}
%	\end{flushright}
%\end{epigrafe}
%%%%%%%%%%%%%%%%%%%%%%%%%%%%%%%%%%%%% ALUNO DEVE EDITAR AQUI %%%%%%%%%%%%%%%%%%%%%%%%%%%%%%%%%%%
% resumo em português
\setlength{\absparsep}{18pt}   % O ALUNO DEVE EDITAR AQUI
\begin{resumo}
	Tenho que fazer um resumo aqui.
 \textbf{Palavras-chaves}:ENEM % O ALUNO DEVE EDITAR AQUI
\end{resumo}
%%%%%%%%%%%%%%%%%%%%%%%%%%%%%%%%%%%%% ALUNO DEVE EDITAR AQUI %%%%%%%%%%%%%%%%%%%%%%%%%%%%%%%%%%%

% resumo em inglês
%\begin{resumo}[Abstract]
% \begin{otherlanguage*}{english}
%   This is the english abstract.     % O ALUNO DEVE EDITAR AQUI
%
%   \vspace{\onelineskip}
% 
%   \noindent 
%   \textbf{Key-words}: latex. abntex. text editoration.    % O ALUNO DEVE EDITAR AQUI
% \end{otherlanguage*}
%\end{resumo}

%%%%%%%%%%%%%%%%%%%%%%%%%%%%%%%%%%%%% ALUNO DEVE EDITAR AQUI %%%%%%%%%%%%%%%%%%%%%%%%%%%%%%%%%%%

% resumo em francês 
%\begin{resumo}[Résumé]
% \begin{otherlanguage*}{french}
%    Il s'agit d'un résumé en français.   % O ALUNO DEVE EDITAR AQUI
% 
%   \textbf{Mots-clés}: latex. abntex. publication de textes.   % O ALUNO DEVE EDITAR AQUI
% \end{otherlanguage*}
%\end{resumo}

%%%%%%%%%%%%%%%%%%%%%%%%%%%%%%%%%%%%% ALUNO DEVE EDITAR AQUI %%%%%%%%%%%%%%%%%%%%%%%%%%%%%%%%%%%

% resumo em espanhol
%\begin{resumo}[Resumen]
% \begin{otherlanguage*}{spanish}
%   Este es el resumen en español.    % O ALUNO DEVE EDITAR AQUI
%  
%   \textbf{Palabras clave}: latex. abntex. publicación de textos.  % O ALUNO DEVE EDITAR AQUI
% \end{otherlanguage*}
%\end{resumo}

%%%%%%%%%%%%%%%%%%%%%%%%%%% ALUNO NÃO DEVE EDITAR %%%%%%%%%%%%%%%%%%%%%%%%%%%%%%%%%%%%%%%%%%%%%%%%

\pdfbookmark[0]{\listfigurename}{lof}
\listoffigures*
\cleardoublepage
% ---
\pdfbookmark[0]{\listtablename}{lot}
\listoftables*
\cleardoublepage
%
%%%%%%%%%%%%%%%%%%%%%%%%%%%%% ALUNO DEVE EDITAR AQUI %%%%%%%%%%%%%%%%%%%%%%%%%%%%%%%%%%%%%%%%
\begin{siglas}   % O ALUNO DEVE EDITAR AQUI
\item[INEP]Instituto Nacional de Estudos e Pesquisas Educacionais Anísio Teixeira
\item[ANDE]Associação Nacional de Educação
\item[ANPEd]Associação Nacional de Pós-Graduação e Pesquisa em Educação
\item[CEDES]Centro de Estudos Educação e Sociedade
\item[CPB]Confederação de Professores do Brasil
\item[CNTE]Confederação Nacional dos Trabalhadores da Educação
\item[ENEM]Exame Nacional do Ensino Médio
\item[SISU]Sistema de Seleção Unificada
\item[SAEB]Sistema de Avaliação da Educação Básica
\end{siglas}
% ---
%%%%%%%%%%%%%%%%%%%%%%%%%%%%% ALUNO DEVE EDITAR AQUI %%%%%%%%%%%%%%%%%%%%%%%%%%%%%%%%%%%%%%%%

\begin{simbolos}   % O ALUNO DEVE EDITAR AQUI
  \item[$ \Gamma $] Letra grega Gama
  \item[$ \Lambda $] Lambda
  \item[$ \zeta $] Letra grega minúscula zeta
  \item[$ \in $] Pertence
\end{simbolos}
% ---
%%%%%%%%%%%%%%%%%%%%%%%%%%% ALUNO NÃO DEVE EDITAR %%%%%%%%%%%%%%%%%%%%%%%%%%%%%%%%%%%%%%%%%%%%%%%%
\pdfbookmark[0]{\contentsname}{toc}
\tableofcontents*
\cleardoublepage
% ---
\textual
%
%%%%%%%%%%%%%%%%%%%% INTRODUÇÃO %%%%%%%%%%%%%%%%%%%%%%%%%%%%%%%%%%%%%%%%%
\chapter[INTRODUÇÃO]{INTRODUÇÃO}   % O ALUNO DEVE EDITAR AQUI

O Instituto Nacional de Estudos e Pesquisas Educacionais Anísio Teixeira (INEP), por intermédio da Diretoria de Avaliação da Educação Básica, em cumprimento da sua missão de desenvolver e disseminar informações sobre os exames e avaliações da educação básica, disponibiliza os Microdados do Enem 2017.

\section{Delimitação do tema}

A educação no Brasil passou por muitas mudanças, principalmente nas ultimas décadas, onde houve por exemplo a expansão no número de instituições (tanto privadas quanto públicas) e ampliação de acesso. O ENEM se consolida, nesse cenário, como principal acesso ao ensino superior. Ele tem também como proposta ser uma ferramenta de autoavaliação das habilidades desenvolvidas pelo aluno ao final do ensino médio. 

O desenvolvimento de tais habilidades pode ser atribuída ao esforço do aluno, porém há de se considerar fatores externos que possam influenciar seu desempenho. Heranças recebidas pelos alunos, não necessariamente monetárias, são variáveis que influenciam o aprendizado segundo \cite{bourdieu1986forms} , sendo elas variáveis latentes. A infraestrutura escolar é um fator importante neste contexto, bem como o contexto social em que o estudante está inserido.

O ENEM, bem como o Censo Escolar fornecem informações necessárias para o presente estudo, sendo levado em consideração o ambiente familiar e escolar. O efeito da turma do aluno, no caso ano série por exemplo, também é um fator influente, porém não temos tal informação a partir dos dados disponíveis.

\section{Objetivos}

O objetivo geral deste trabalho é analisar quais variáveis socioeconômicas influenciaram os estudantes que fizeram o ENEM no estado do Paraná no ano de 2017. Para tal será utilizado um modelo multinível multivariado onde as variáveis respostas serão as notas nas 4 áreas do ENEM. Como variáveis explicativas de primeiro nível, serão utilizados informações socioeconômicas encontradas no ENEM bem como sua pontuação em habilidades relacionadas à redação, e como variáveis de segundo nível será utilizada informações sobre a escola do aluno obtidas pelo Censo escolar 2017.
  
%
	%%%%%%%%%%%%%%%%%%%%%%% REVISÃO BIBLIOGRAFICA %%%%%%%%%%%%%%%%%%%%%%%% 
	\chapter{REVISÃO BIBLIOGRÁFICA}

	Neste capitulo, busca-se contextualizar o ENEM tanto como avaliação do desempenho individual do aluno, bem como ferramenta pedagógica, visto o desenvolvimento educacional brasileiro ao longo da história bem como a própria proposta da prova.

	\section{Sobre história da educação}

	A educação no Brasil se inicia com a chegada dos jesuítas como primeiros educadores, a partir de 1549. \citeonline{saviani2007historia} define esse como o primeiro período educacional no Brasil, marcado pelos princípios colonizador e catequizador, com ênfase nos povos indígenas. Neste período, os colonizadores ao tentar escravizar indígenas, foram confrontados pelos jesuítas.

	No segundo período, a fim de acabar com tal embate, Marquês de Pombal determina a expulsão dos jesuítas do território nacional na segunda metade do século XVIII, reduzindo a influência dos padres no ensino brasileiro. Neste momento, é fomentada a discussão sobre propostas de ensino, tendo os métodos intuitivo e o mútuo ganhado espaço. A Lei de 15 de outubro de 1827 cria as "escolas de primeiras letras" em todas as cidades, vilas e lugares mais populosos do Império, onde tais métodos passam a ser utilizados.

	No período seguinte, buscou-se o equilíbrio entre a pedagogia tradicional e a pedagogia nova. Pensadores educacionais começam a olhar o indivíduo como sendo o foco do aprendizado, culminando no \textbf{Manifesto dos Pioneiros da Educação Nova} . Neste período, é encaminhado o primeiro projeto sobre as Lei de Diretrizes e Bases da Educação Nacional ao Congresso Nacional, sendo aprovado apenas na década de 1960.

	No quarto período, junto a ditadura militar, foi difundida a pedagogia tecnicista, a qual privilegiava excessivamente a tecnologia educacional e transformava professores e alunos em meros executores e receptores de projetos elaborados de forma autoritária e sem qualquer vínculo com o contexto social a que se destinavam. Porém, é neste período que a pedagogia crítica ganha força. Baseada na teoria marxista, ela objetiva habilitar o estudante a pensar criticamente quanto a sua realidade, sendo suas ideias opostas às do regime militar. Nesta época também nascem organizações como a Associação Nacional de Educação (ANDE), Associação Nacional de Pós-Graduação e Pesquisa em Educação (ANPEd), Centro de Estudos Educação e Sociedade (CEDES), fortalecendo a Confederação de Professores do Brasil (CPB), que, posteriormente, torna-se a Confederação Nacional dos Trabalhadores da Educação (CNTE). Tais associações docentes, apoiando-se no momento nacional, passam a estabelecer laços sindicais. Tudo isso ajuda a fortalecer a produção científica preocupada com "a construção de uma escola pública de qualidade" \citeonline[p. 402]{saviani2007historia}. Também nesta época, com a transição do fordismo para o toyotismo, as ideias pedagógicas no Brasil se manifestam no neoprodutivismo, tratando o conhecimento como um bem do aluno na forma de capital humano, o que acaba se transformando na "pedagogia da exclusão". O neoescolanovismo reascende a ideia de "aprender a aprender", sendo essa uma atividade construtiva do aluno. O Estado passa a tentar maximizar seus resultados educacionais reorganizando suas formas de ensino.

	\section{O ENEM}

	Criado em 1998, o Exame Nacional do Ensino Médio (Enem) tem como objetivo verificar se seus participantes apresentam domínio dos princípios científicos e tecnológicos ao final do ensino médio, domínio dos princípios científicos e tecnológicos que presidem a produção moderna e se detêm conhecimento das formas contemporâneas de linguagem. 

	Seus resultados deverão possibilitar: 

	I - a constituição de parâmetros para a autoavaliação do participante, com vistas à continuidade de sua formação e a sua inserção no mercado de trabalho; 

	II - a criação de referência nacional para o aperfeiçoamento dos currículos do ensino médio; 

	III - a utilização do Exame como mecanismo único, alternativo ou complementar para acesso à educação superior, especialmente a ofertada pelas instituições federais de educação superior;

	IV - o acesso a programas governamentais de financiamento ou apoio ao estudante da educação;

	V - a sua utilização como instrumento de seleção para ingresso nos diferentes setores do mundo do trabalho;

	VI - o desenvolvimento de estudos e indicadores sobre a educação brasileira. 

	O exame já era utilizado por algumas instituições de Ensino Superior como instrumento de seleção para o ingresso de estudantes, quando em 2009 passou pela reformulação a fim de democratizar o acesso. A partir de então muitas outras universidades federais passaram a utilizá-lo como método de ingresso através do SISU. Reformularam-se as Matrizes de Referência do Exame, que passaram a ser estruturadas por competências em quatro áreas do conhecimento – Linguagens, Códigos e suas Tecnologias, Matemática e suas Tecnologias, Ciências Humanas e suas Tecnologias e Ciências da Natureza e suas Tecnologias –, apresentando, ainda, eixos cognitivos comuns a todas as áreas. Essa mesma reformulação permitiu que obedecendo a Lei de Diretrizes e Bases da Educação Nacional (Lei 9.394/1996) \nocite{educaccao1996lei}, alunos pudessem utilizar a prova como certificação de conclusão do Ensino Médio, porém em 2017 a prova deixou de ter esta atribuição.

	\begin{table}[]
	\caption { Descrição das Áreas de Conhecimento e Componentes Curriculares do Enem.} 
	\begin{tabular}{|l|p{8 cm}|}
	\hline
	Área do conhecimento & Componentes Curriculares \\ \hline
	Linguagens, Códigos e suas tecnologias     &   Língua Portuguesa, Literatura, Língua Estrangeira (Inglês ou Espanhol), Artes, Educação Física e Tecnologias da Informação e Comunicação.\\
		        Ciências Humanas e suas tecnologias                           &     História, Geografia, Filosofia e Sociologia. \\
		         Ciências da Natureza e suas tecnologias                           &     Química, Física e Biologia.  \\
		         Matemática e suas tecnologias  & Matemática. \\                                        \hline
	\end{tabular}
	\begin{tablenotes}
	      \small
	      \item \textbf{Nota Fonte}  Adaptado do Enem (2017)
	    \end{tablenotes}
	\end{table}

	Houve ainda uma segunda aplicação do Enem 2017, ocorrendo nos dias 12 e 13 de dezembro de 2017. Esta data foi destinada para Pessoas Privadas de Liberdade e Jovens sob Medida Socioeducativa que inclua privação de liberdade – PPL, e também a participantes com direito à reaplicação.

	Neste ano além das provas objetivas e da redação, os participantes do Enem respondem a um questionário que contempla questões sobre seu nível socioeconômico, família, educação e trabalho, havendo uma diminuição no número de questões do questionário, contemplando 27 questões no total.

	\section{Estudos sobre o ENEM}

	\citeonline{figueiredo2014igualdade} investigaram como o background de um indivíduo que realiza ENEM influência em seu desempenho. Para isso, consideraram variáveis como a renda familiar, a escolaridade dos pais, se estudou em escola pública ou privada, a raça, se mora na área urbana ou rural e o índice de qualidade escolar do SAEB.

	Duas abordagens foram tomadas: a primeira assume que o esforço do indivíduo e as circunstâncias em que ele se encontra, ou em que as variáveis foram mensuradas, são independentes. A segunda relaxa essas hipóteses, ou seja, considera que pode existir uma relação entre essas quantidades. Esses modelos foram baseados em estudos realizados anteriormente por Roemer(1998) e Hanushek(1970, 1979, 2007).

	As técnicas utilizadas para mensurar o assunto são baseadas no modelo de classificação não paramétrico, Axioma de Identificação de Roemer, proposto por Roemer e na função de produção educacional, de Hanushek.

	\citeonline{figueiredo2014igualdade} concluíram que as variáveis socioeconômicas consideradas influenciam diretamente no esforço do indivíduo e, consequentemente na sua pontuação. Indivíduos de “baixo background” têm de se esforçar 99,38 \% mais do que aqueles que possuem “alto background” para terem as mesmas chances de sucesso no exame. Por isso, parece inviável que aqueles participantes penalizados pelas circunstâncias obtenham as melhores notas. Esses resultados são ainda mais extremos na Região Nordeste.

	\citeonline{nascimento2018pursuit} se baseiam nas ideias de Pierre Bourdieu e descrevem que o aprendizado é influenciado por heranças culturais advindas do convívio social , constituem o que Bourdieu chama de capital cultural, capital econômico (bens materiais, propriedades e outros) e o capital social (relações sociais com pessoas chave, prestígio, poder político). Capitais culturais seriam o que a pessoa possui que não é mensurado em objetos ou dinheiro, seriam mais relacionados a traços psicológicos, sendo tais que identifica três estados nos quais esse capital pode ser encontrado: institucionalizado, objetivado e incorporado. . O estado institucionalizado corresponde aos títulos escolares acumulados pelo estudante. O estado objetivado consiste dos bens culturais que o sujeito tem acesso, como livros, revistas, enciclopédias e obras de arte, por exemplo. O mais importante dos estados certamente é o incorporado. A partir desta perspectiva o autor espera encontrar que as variáveis mais importantes estão associados aos capitais culturais incorporados.

	Começam separando os alunos em níveis socioeconômicos através da análise fatorial completa, onde permite separar as questões socioeconômicas em 3 fatores(que coincidiram com os propostos por Bourdieu) e usar o peso destes fatores para representar os estados, dentre os quais foi utilizado o correspondente ao capital cultural institucionalizado, pois está mais correlacionado com a nota. A partir de tais variáveis, foi utilizada a análise de cluster para se construir níveis socioeconômicos através do método de K-means bivariada.

	Para analisar a dificuldade da questão, os autores usam um modelo logístico de 3 variáveis, para determinar a dificuldade da questão.

%
%%%%%%%%%%%%%%%%%%%%%%%%% DESCRIÇÃO DOS DADOS %%%%%%%%%%%%%%%%%%%
\chapter[BASE DE DADOS]{BASE DE DADOS} % CAIXA ALTA

Nesta seção, apresentam-se a descrição das fontes de dados e o tratamento aplicado à base obtida.

\section{Descrição das fontes de dados}

As bases de dados utilizadas foram os microdados do ENEM e os microdados do censo escolar, disponibilizados pelo Instituto Nacional de Estudos e Pesquisas Educacionais Anísio Teixeira (INEP),sendo o primeiro por intermédio da Diretoria de Avaliação da Educação Básica. Foi tomado 2017 como ano de referência, pois até o inicio deste trabalho ainda não havia sido disponibilizado os dados referentes ao ENEM 2018.
Dos microdados do ENEM foram selecionadas as variáveis:
\begin{itemize}
	\item NU\_IDADE - idade do aluno;
	\item TP\_SEXO - sexo do aluno (masculino; feminino);
	\item TP\_ESTADO\_CIVIL - estado civil do aluno (Solteiro(a); Casado(a)/Mora com companheiro(a); Divorciado(a)/Desquitado(a)/Separado(a);  Viúvo(a));
	\item TP\_COR\_RACA - cor ou raça do aluno (Não declarado; Branca; Preta; Parda; Amarela; Indígena);
		\item TP\_ST\_CONCLUSAO - Situação de conclusão do Ensino Médio (Já concluí o Ensino Médio; Estou cursando e concluirei o Ensino Médio em 201;Estou cursando e concluirei o Ensino Médio após 2017; Não concluí e não estou cursando o Ensino Médio);
		\item CO\_ESCOLA -  Número gerado como identificador da escola no Censo Escolar da Educação Básica;
		\item NU\_NOTA\_MT - Nota da prova de Matemática;
		\item Q001 - Até que série o pai, ou o homem responsável por pelo aluno, estudou (Nunca estudou; Não completou a 4ª série/5º ano do Ensino Fundamental; Completou a 4ª série/5º ano, mas não completou a 8ª série/9º ano do Ensino Fundamental; Completou a 8ª série/9º ano do Ensino Fundamental, mas não completou o Ensino Médio; Completou o Ensino Médio, mas não completou a Faculdade;Completou a Faculdade, mas não completou a Pós-graduação;Completou a Pós-graduação;Não sei);
		\item Q002 - Até que série a mãe, ou a mulher responsável por pelo aluno, estudou (Nunca estudou; Não completou a 4ª série/5º ano do Ensino Fundamental; Completou a 4ª série/5º ano, mas não completou a 8ª série/9º ano do Ensino Fundamental; Completou a 8ª série/9º ano do Ensino Fundamental, mas não completou o Ensino Médio; Completou o Ensino Médio, mas não completou a Faculdade;Completou a Faculdade, mas não completou a Pós-graduação;Completou a Pós-graduação;Não sei);
		\item Q008 - Na residência tem banheiro( Não; Sim, um; Sim, dois; Sim, três;Sim, quatro ou mais);
		\item Q025 - Na residência tem acesso à Internet (Não; Sim);
	\end{itemize}
	No microdados do censo escolar, foram selecionadas as variáveis:
	\begin{itemize}
		\item CO\_MUNICIPIO - Código do Município;
		\item CO\_DISTRITO - Código completo do Distrito da escola;
		\item TP\_LOCALIZACAO - Localização (Urbana; Rural);
		\item IN\_AGUA\_FILTRADA - Água consumida pelos alunos na escola passa por um processo de filtragem (Não filtrada; Filtrada);
		\item IN\_AGUA\_REDE\_PUBLICA - Abastecimento de água através de rede pública;
		\item IN\_AGUA\_POCO\_ARTESIANO - Abastecimento de água através de poço artesiano;
		\item IN\_AGUA\_CACIMBA - Abastecimento de água através de cacimba/Cisterna/Poço;
		\item IN\_AGUA\_FONTE\_RIO - Abastecimento de água através de Fonte/Rio/Igarapé/Riacho/Córrego;
		\item IN\_AGUA\_INEXISTENTE - Abastecimento de água inexistente;
		\item IN\_BIBLIOTECA\_SALA\_LEITURA - Existe biblioteca ou sala de leitura nas dependências da escola;
		\item IN\_BANHEIRO\_FORA\_PREDIO - O banheiro se encontra fora da escola;
		\item IN\_BANHEIRO\_DENTRO\_PREDIO - O banheiro se encontra dentro do prédio;
		\item IN\_LIXO\_COLETA\_PERIODICA - Lixo é coletado periodicamente;
		\item IN\_LIXO\_RECICLA - O lixo é reciclado;
		\item IN\_BANHEIRO\_PNE - Banheiro das dependências da escola adequado ao uso dos alunos com deficiência ou mobilidade reduzida;
		\item IN\_SECRETARIA - Dependências existentes na escola sala de secretaria;
		\item IN\_ENERGIA\_INEXISTENTE - Abastecimento de energia elétrica inexistente;
		\item IN\_INTERNET - Acesso à Internet;
		\item IN\_BANDA\_LARGA - Internet Banda Larga;
		\item IN\_QUADRA\_ESPORTES\_COBERTA - Dependências existentes na escol, quadra de esportes Coberta;
		\item NU\_EQUIP\_TV - Quantidade de Aparelhos de televisão;
		\item NU\_EQUIP\_DVD - Quantidade de Aparelhos de DVD;
		\item IN\_SALA\_PROFESSOR - Dependências existentes na escola sala de professores;
		\item IN\_EQUIP\_COPIADORA - Equipamentos existentes na escola copiadora;
		\item IN\_LABORATORIO\_INFORMATICA - Dependências existentes na escola laboratório de informática;
		\item IN\_LABORATORIO\_CIENCIAS - Dependências existentes na escola laboratório de ciências;
		\item IN\_EQUIP\_IMPRESSORA - Equipamentos existentes na escola impressora;
		\item IN\_COMPUTADOR - Equipamentos existentes na escola computador;
		\item TP\_AEE - Atendimento Educacional Especializado (AEE);
		\item IN\_BIBLIOTECA -  Dependências existentes na escola biblioteca;
		\item IN\_PATIO\_COBERTO - Dependências existentes na escola pátio Coberto;
	\end{itemize}
	As variáveis de segundo nível foram escolhidas seguindo a proposta de \citeonline{thiago}. As variáveis sobre escolaridade dos responsáveis foram escolhidas por estarem em conformidade com as ideias de \citeonline{bourdieu1986forms}
\section{Tratamento das bases de dados}

%
%%%%%%%%%%%%%%%%%%%%%%%%%%%%% DESENVOLVIMENTO %%%%%%%%%%%%%%%%%%%%%%%%%%%%%%%%%%%%%%%%%
\chapter{MODELAGEM ESTATÍSTICA} % CAIXA ALTA
% ---
\section{Análise exploratória dos dados}

\section{Definição do modelo proposto}

%
%%%%%%%%%%%%%%%%% METODOLOGIA%%%%%%%%%%%%%%%%%%%%%%%%%%%%%%%%%%%%%%%%%%%%%%%%%%%%
\chapter{METODOLOGIA} % CAIXA ALTA
\chapterprecis{Referencial teórico sobre os métodos quantitativos aplicados.}\index{sinopse de capítulo}

% ---
% ---



% 
%%%%%%%%%%%%%%%%%%%%%%%%% ANÁLISE DOS RESULTADOS %%%%%%%%%%%%%%%%%%%%%%%%%%%%%%%%%%%%
\chapter{ANÁLISE DOS RESULTADOS} % CAIXA ALTA


%
\bookmarksetup{startatroot}% 
% 
%%%%%%%%%%%%%%%%%%%%%%%%%% CONCLUSÃO %%%%%%%%%%%%%%%%%%%%%%%%%%%%%%%%%%%%%%%%%
\chapter[CONCLUSÃO]{CONCLUSÃO} % CAIXA ALTA

\chapterprecis{Comentários finais e trabalhos futuros.}\index{sinopse de capítulo}
%
\postextual
%
\bibliography{ref}
%
% ----------------------------------------------------------
% Glossário
% ----------------------------------------------------------
%Capítulo excluído
% Consulte o manual da classe abntex2 para orientações sobre o glossário.
%
%\glossary
% ---
%\begin{apendicesenv}
%%
%\partapendices
%%%%%%%%%%%%%%%%%%%%%%%%%%%%%%%%%%%%%%%%%%%%% ALUNO DEVE EDITAR AQUI APENDICES %%%%%%%%%%%%%%%%%%%%%%%%%%%%%%%%%%%%%%%%
%% ----------------------------------------------------------
%\chapter{Capitulo}
%% ----------------------------------------------------------
%\chapter{Caplitulo 2}
%% ----------------------------------------------------------
%\end{apendicesenv}
%% 
%\begin{anexosenv}
%%
%\partanexos
%%%%%%%%%%%%%%%%%%%%%%%%%%%%%%%%%%%%%%%%%%%%%% ALUNO DEVE EDITAR AQUI ANEXOS %%%%%%%%%%%%%%%%%%%%%%%%%%%%%%%%%%%%%%%%%%%%
%\chapter{Capitulo}
%% ---
%\chapter{Capitulo}
%%
%\chapter{Capitulo}
%% 
%\end{anexosenv}
%%%%%%%%%%%%%%%%%%%%%%%%%%%%%%%%%%%%%%%%%%% NÃO EDITAR %%%%%%%%%%%%%%%%%%%%%%%%%%%%%%%%%%%%%%%%%%%%%%%%%%%%%%%%%%%%%%%%%
%---------------------------------------------------------------------
% INDICE REMISSIVO
%---------------------------------------------------------------------
\printindex
%
\end{document}
