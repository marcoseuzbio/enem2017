%%%%%%%%%%%%%%%%%%%% INTRODUÇÃO %%%%%%%%%%%%%%%%%%%%%%%%%%%%%%%%%%%%%%%%%
\chapter[INTRODUÇÃO]{INTRODUÇÃO}   % O ALUNO DEVE EDITAR AQUI

O Instituto Nacional de Estudos e Pesquisas Educacionais Anísio Teixeira (INEP), por intermédio da Diretoria de Avaliação da Educação Básica, em cumprimento da sua missão de desenvolver e disseminar informações sobre os exames e avaliações da educação básica, disponibiliza os Microdados do Enem 2017.

\section{Delimitação do tema}

A educação no Brasil passou por muitas mudanças, principalmente nas ultimas décadas, onde houve por exemplo a expansão no número de instituições (tanto privadas quanto públicas) e ampliação de acesso. O ENEM se consolida, nesse cenário, como principal acesso ao ensino superior. Ele tem também como proposta ser uma ferramenta de autoavaliação das habilidades desenvolvidas pelo aluno ao final do ensino médio. 

O desenvolvimento de tais habilidades pode ser atribuída ao esforço do aluno, porém há de se considerar fatores externos que possam influenciar seu desempenho. Heranças recebidas pelos alunos, não necessariamente monetárias, são variáveis que influenciam o aprendizado segundo \citeonline{bourdieu1986forms} , sendo elas variáveis latentes. A infraestrutura escolar é um fator importante neste contexto, bem como o contexto social em que o estudante está inserido.

O ENEM, bem como o Censo Escolar fornecem informações necessárias para o presente estudo, sendo levado em consideração o ambiente familiar e escolar. O efeito da turma do aluno, no caso ano série por exemplo, também é um fator influente, porém não temos tal informação a partir dos dados disponíveis.

\section{Objetivos}

O objetivo geral deste trabalho é analisar quais variáveis socioeconômicas influenciaram os estudantes que fizeram o ENEM no estado do Paraná no ano de 2017. Para tal será utilizado um modelo multinível multivariado onde as variáveis respostas serão as notas nas 4 áreas do ENEM. Como variáveis explicativas de primeiro nível, serão utilizados informações socioeconômicas encontradas no ENEM bem como sua pontuação em habilidades relacionadas à redação, e como variáveis de segundo nível será utilizada informações sobre a escola do aluno obtidas pelo Censo escolar 2017.
