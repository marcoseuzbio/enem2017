%%%%%%%%%%%%%%%%%%%%%%% REVISÃO BIBLIOGRÁFICA %%%%%%%%%%%%%%%%%%%%%%%% 
\chapter{REVISÃO BIBLIOGRÁFICA}

	Neste capitulo, busca-se contextualizar o ENEM tanto como avaliação do desempenho individual do aluno, bem como ferramenta pedagógica, visto o desenvolvimento educacional brasileiro ao longo da história bem como a própria proposta da prova.

\section{Sobre história da educação}

	A educação no Brasil se inicia com a chegada dos jesuítas como primeiros educadores, a partir de 1549. \citeonline{saviani2007historia} define esse como o primeiro período educacional no Brasil, marcado pelos princípios colonizador e catequizador, com ênfase nos povos indígenas. Neste período, os colonizadores ao tentar escravizar indígenas, foram confrontados pelos jesuítas.

	No segundo período, a fim de acabar com tal embate, Marquês de Pombal determina a expulsão dos jesuítas do território nacional na segunda metade do século XVIII, reduzindo a influência dos padres no ensino brasileiro. Neste momento, é fomentada a discussão sobre propostas de ensino, tendo os métodos intuitivo e o mútuo ganhado espaço. A Lei de 15 de outubro de 1827 cria as "escolas de primeiras letras" em todas as cidades, vilas e lugares mais populosos do Império, onde tais métodos passam a ser utilizados.

	No período seguinte, buscou-se o equilíbrio entre a pedagogia tradicional e a pedagogia nova. Pensadores educacionais começam a olhar o indivíduo como sendo o foco do aprendizado, culminando no \textbf{Manifesto dos Pioneiros da Educação Nova} . Neste período, é encaminhado o primeiro projeto sobre as Lei de Diretrizes e Bases da Educação Nacional ao Congresso Nacional, sendo aprovado apenas na década de 1960.

	No quarto período, junto a ditadura militar, foi difundida a pedagogia tecnicista, a qual privilegiava excessivamente a tecnologia educacional e transformava professores e alunos em meros executores e receptores de projetos elaborados de forma autoritária e sem qualquer vínculo com o contexto social a que se destinavam. Porém, é neste período que a pedagogia crítica ganha força. Baseada na teoria marxista, ela objetiva habilitar o estudante a pensar criticamente quanto a sua realidade, sendo suas ideias opostas às do regime militar. Nesta época também nascem organizações como a Associação Nacional de Educação (ANDE), Associação Nacional de Pós-Graduação e Pesquisa em Educação (ANPEd), Centro de Estudos Educação e Sociedade (CEDES), fortalecendo a Confederação de Professores do Brasil (CPB), que, posteriormente, torna-se a Confederação Nacional dos Trabalhadores da Educação (CNTE). Tais associações docentes, apoiando-se no momento nacional, passam a estabelecer laços sindicais. Tudo isso ajuda a fortalecer a produção científica preocupada com "a construção de uma escola pública de qualidade" \citeonline[p. 402]{saviani2007historia}. Também nesta época, com a transição do fordismo para o toyotismo, as ideias pedagógicas no Brasil se manifestam no neoprodutivismo, tratando o conhecimento como um bem do aluno na forma de capital humano, o que acaba se transformando na "pedagogia da exclusão". O neoescolanovismo reascende a ideia de "aprender a aprender", sendo essa uma atividade construtiva do aluno. O Estado passa a tentar maximizar seus resultados educacionais reorganizando suas formas de ensino.

	\section{O ENEM}

	Criado em 1998, o Exame Nacional do Ensino Médio (Enem) tem como objetivo verificar se seus participantes apresentam domínio dos princípios científicos e tecnológicos ao final do ensino médio, domínio dos princípios científicos e tecnológicos que presidem a produção moderna e se detêm conhecimento das formas contemporâneas de linguagem. 

	Seus resultados deverão possibilitar: 

	I - a constituição de parâmetros para a autoavaliação do participante, com vistas à continuidade de sua formação e a sua inserção no mercado de trabalho; 

	II - a criação de referência nacional para o aperfeiçoamento dos currículos do ensino médio; 

	III - a utilização do Exame como mecanismo único, alternativo ou complementar para acesso à educação superior, especialmente a ofertada pelas instituições federais de educação superior;

	IV - o acesso a programas governamentais de financiamento ou apoio ao estudante da educação;

	V - a sua utilização como instrumento de seleção para ingresso nos diferentes setores do mundo do trabalho;

	VI - o desenvolvimento de estudos e indicadores sobre a educação brasileira. 

	O exame já era utilizado por algumas instituições de Ensino Superior como instrumento de seleção para o ingresso de estudantes, quando em 2009 passou pela reformulação a fim de democratizar o acesso. A partir de então muitas outras universidades federais passaram a utilizá-lo como método de ingresso através do SISU. Reformularam-se as Matrizes de Referência do Exame, que passaram a ser estruturadas por competências em quatro áreas do conhecimento – Linguagens, Códigos e suas Tecnologias, Matemática e suas Tecnologias, Ciências Humanas e suas Tecnologias e Ciências da Natureza e suas Tecnologias –, apresentando, ainda, eixos cognitivos comuns a todas as áreas. Essa mesma reformulação permitiu que obedecendo a Lei de Diretrizes e Bases da Educação Nacional (Lei 9.394/1996) \nocite{educaccao1996lei}, alunos pudessem utilizar a prova como certificação de conclusão do Ensino Médio, porém em 2017 a prova deixou de ter esta atribuição.

	\begin{table}[]
	\caption { Descrição das Áreas de Conhecimento e Componentes Curriculares do Enem.} 
	\begin{tabular}{|l|p{8 cm}|}
	\hline
	Área do conhecimento & Componentes Curriculares \\ \hline
	Linguagens, Códigos e suas tecnologias     &   Língua Portuguesa, Literatura, Língua Estrangeira (Inglês ou Espanhol), Artes, Educação Física e Tecnologias da Informação e Comunicação.\\
		        Ciências Humanas e suas tecnologias                           &     História, Geografia, Filosofia e Sociologia. \\
		         Ciências da Natureza e suas tecnologias                           &     Química, Física e Biologia.  \\
		         Matemática e suas tecnologias  & Matemática. \\                                        \hline
	\end{tabular}
	\begin{tablenotes}
	      \small
	      \item \textbf{Nota Fonte}  Adaptado do Enem (2017)
	    \end{tablenotes}
	\end{table}

	Houve ainda uma segunda aplicação do Enem 2017, ocorrendo nos dias 12 e 13 de dezembro de 2017. Esta data foi destinada para Pessoas Privadas de Liberdade e Jovens sob Medida Socioeducativa que inclua privação de liberdade – PPL, e também a participantes com direito à reaplicação.

	Neste ano além das provas objetivas e da redação, os participantes do Enem respondem a um questionário que contempla questões sobre seu nível socioeconômico, família, educação e trabalho, havendo uma diminuição no número de questões do questionário, contemplando 27 questões no total.

	\section{Estudos sobre o ENEM}

	\citeonline{figueiredo2014igualdade} investigaram como o background de um indivíduo que realiza ENEM influência em seu desempenho. Para isso, consideraram variáveis como a renda familiar, a escolaridade dos pais, se estudou em escola pública ou privada, a raça, se mora na área urbana ou rural e o índice de qualidade escolar do SAEB.

	Duas abordagens foram tomadas: a primeira assume que o esforço do indivíduo e as circunstâncias em que ele se encontra, ou em que as variáveis foram mensuradas, são independentes. A segunda relaxa essas hipóteses, ou seja, considera que pode existir uma relação entre essas quantidades. Esses modelos foram baseados em estudos realizados anteriormente por Roemer(1998) e Hanushek(1970, 1979, 2007).

	As técnicas utilizadas para mensurar o assunto são baseadas no modelo de classificação não paramétrico, Axioma de Identificação de Roemer, proposto por Roemer e na função de produção educacional, de Hanushek.

	\citeonline{figueiredo2014igualdade} concluíram que as variáveis socioeconômicas consideradas influenciam diretamente no esforço do indivíduo e, consequentemente na sua pontuação. Indivíduos de “baixo background” têm de se esforçar 99,38 \% mais do que aqueles que possuem “alto background” para terem as mesmas chances de sucesso no exame. Por isso, parece inviável que aqueles participantes penalizados pelas circunstâncias obtenham as melhores notas. Esses resultados são ainda mais extremos na Região Nordeste.

	\citeonline{nascimento2018pursuit} se baseiam nas ideias de Pierre Bourdieu e descrevem que o aprendizado é influenciado por heranças culturais advindas do convívio social , constituem o que Bourdieu chama de capital cultural, capital econômico (bens materiais, propriedades e outros) e o capital social (relações sociais com pessoas chave, prestígio, poder político). Capitais culturais seriam o que a pessoa possui que não é mensurado em objetos ou dinheiro, seriam mais relacionados a traços psicológicos, sendo tais que identifica três estados nos quais esse capital pode ser encontrado: institucionalizado, objetivado e incorporado. . O estado institucionalizado corresponde aos títulos escolares acumulados pelo estudante. O estado objetivado consiste dos bens culturais que o sujeito tem acesso, como livros, revistas, enciclopédias e obras de arte, por exemplo. O mais importante dos estados certamente é o incorporado. A partir desta perspectiva o autor espera encontrar que as variáveis mais importantes estão associados aos capitais culturais incorporados.

	Começam separando os alunos em níveis socioeconômicos através da análise fatorial completa, onde permite separar as questões socioeconômicas em 3 fatores(que coincidiram com os propostos por Bourdieu) e usar o peso destes fatores para representar os estados, dentre os quais foi utilizado o correspondente ao capital cultural institucionalizado, pois está mais correlacionado com a nota. A partir de tais variáveis, foi utilizada a análise de cluster para se construir níveis socioeconômicos através do método de K-means bivariada.

	Para analisar a dificuldade da questão, os autores usam um modelo logístico de 3 variáveis, para determinar a dificuldade da questão.

\citeonline{thiago} propôs uma modelagem hierárquica, como de verificar quais variáveis socioeconômicas influenciaram os estudantes no ENEM de 2014 no estado do Rio de Janeiro. Ele afirma que o desempenho do aluno apresenta relação com a infraestrutura escolar. Seu modelo considera o aluno como variáveis de primeiro nível e a escola como variáveis de segundo nível, apoiando-se na proposta de \citeonline{neto}.
As variáveis selecionadas foram:
\begin{itemize}
\item A diferença entre a renda per capita estimada do aluno pela renda média da escola.
\item Variável indicando se o pai ou a mãe possuam pós-graduação ou não.
\item Variável indicando se o pai ou a mãe possuem como grau de instrução máximo graduação completa/incompleta ou não.
\item Variável indicadora se o pai ou a mãe possuem como grau de instrução máximo ensino médio completo.
\item Variável indicando se o aluno trabalhava na época em que fez a prova.
\item Variável indicadora se o aluno seja do gênero masculino.
\item Escore de infraestrutura da escola que o aluno estudava quando fez a prova do ENEM. Para calcular este escore, o autor gerou um modelo TRI com as variáveis:
\begin{itemize}
\item Água Filtrada para os alunos.
\item Banheiro dentro do prédio.
\item Água - Rede pública.
\item Água - Rede pública ou poço artesiano.
\item Água - Rede pública ou Poço artesiano ou Cacimba.
\item Água Rede pública ou Poço artesiano ou Cacimba ou Fonte/rio.
\item Energia Elétrica.
\item Esgoto Sanitário.
\item Lixo - Coleta Periódica.
\item Lixo - Coleta Periódica de Lixo ou Reciclagem.
\item Sala da Diretoria.
\item Sala dos Professores.
\item Laboratório de Informática.
\item Laboratório de Ciências.
\item Sala de Atendimento Especial (AEE).
\item Biblioteca.
\item Banheiro dentro ou fora do prédio.
\item Banheiro adequado a alunos com deficiência ou mobilidade reduzida.
\item Dependências PNE.
\item Secretaria.
\item Acesso à internet.
\item Acesso à internet banda larga.
\item Quadra de esportes coberto ou descoberto.
\item Pelo menos uma TV.
\item Pelo menos um DVD.
\item Pelo menos uma copiadora.
\item Pelo menos uma impressora.
\item Pelo menos um computador para os alunos.
\item Computadores suficientes.
\item Pátio coberto ou descoberto.
\end{itemize}
\item Variáveis indicadoras se as escolas que os alunos estudaram fossem de dependência administrativa federal, privada ou municipal.
\item Renda média dos alunos na escola.
\end{itemize}

